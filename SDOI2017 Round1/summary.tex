%-*-    coding: UTF-8   -*-
\documentclass[UTF-8]{ctexart}
%begin之前的部分称为导言区,一般对文档进行一些设置或定义一些命令。

\usepackage{graphicx}
\usepackage{subfigure}
\usepackage{amsmath}
\usepackage{tabularx}%表格扩展
\usepackage{color}
\usepackage{xcolor}
\usepackage{hyperref}
\usepackage{ulem}
\usepackage{multirow}%表格合并
\usepackage{amsmath}%数学公式
\usepackage{longtable}%表格跨页
\usepackage[cache=false]{minted}%代码高亮
%\usepackage{fontspec}%字体设置
\usepackage{fontspec,xunicode,xltxtra}
\usemintedstyle{vim}%代码高亮主题
%取消链接的红框
\hypersetup{
	colorlinks=true,
	linkcolor=black
}

%设计页面尺寸,使用geometry宏包
%此处使用A4纸大小,版心居中,长宽占页面的0.8
\usepackage{geometry}
\geometry{a4paper,centering,scale=0.8}
%改变图表标题格式,此处使用悬挂对齐方式(编号向左突出),小字号,标题使用斜体
\usepackage[format=hang,font=small,textfont=it]{caption}
%增加目录项目,tocbibind宏包会自动加入目录项本身、参考文献、索引等项目。[nottoc]取消了对自身的显示
\usepackage[nottoc]{tocbibind}

\definecolor{mintedbg}{RGB}{11,23,70}
%\setmainfont{Ubuntu Mono}
%\setsansfont{Ubuntu Mono}
%\setmonofont{Ubuntu Mono}

%下面三行title、author和date,其通过\maketitle输出到文档
\title{SDOI2017第一轮题解}
\author{MLEAutoMaton}
\date{\today}

%\bibliographystyle声明参考文献的格式
\bibliographystyle{plain}
\def\wholeart{test}
\begin{document}
	\maketitle%输出题目
	\newpage
	\tableofcontents%输出目录
	\newpage%换页
	
	\section{数字表格(product)}
	\subsection{30pts}
	直接暴力枚举每一个点然后计算对应的斐波那契值,全部乘起来就是答案。
	
	\subsection{100pts}
	我们考虑题目要求的式子的化简,这里规定$n < m$。
	\begin{align}
		\prod_{i=1}^n\prod_{j=1}^mf(gcd(i,j))&=\prod_{i=1}^n\prod_{j=1}^m\sum_{d}[gcd(i,j)=d]*f(d)\\
		&=\prod_{d=1}^nf(d)^{\sum_{i=1}^n\sum_{j=1}^m[gcd(i,j)=d]}\\
		&=\prod_{d=1}^nf(d)^{\sum_{x=1}^{n/d}\mu(x)*\lfloor \frac{n}{xd}\rfloor\lfloor \frac{m}{xd}\rfloor}\\
		&=\prod_{T=1}^n\prod_{d|T} f(d)^{\mu(\frac{T}{d})^{\lfloor \frac{n}{T}\rfloor\lfloor \frac{m}{T}\rfloor}}
	\end{align}
	
	此时我们把$\prod_{d|T} f(d)^{\mu(\frac{T}{d})}$这部分预处理,后面的整除分块就解决了这个问题。
	\section{树点涂色(paint)}
	\subsection{10pts}
	直接暴力修改即可。
	\subsection{100pts}
	我们不难发现,维护从一个点到根的路径的颜色修改等于说是$LCT$里面的$access$操作对吧。
	
	这时我们可以发现,一次实链修改等同于是对一个子树$+1$,对一个子树$-1$。
	
	所以我们在$LCT$的$access$里面修改即可。
	
	剩下两个操作直接线段树+树链剖分即可。
	
	代码不是特别长。
	\section{序列计数(count)}
	\subsection{吐槽}
	部分分给的最良心的一道题目。
	\subsection{20pts}
	首先将问题转换成一个容斥的形式,把质数的限制去掉。
	
	显然有一个$\theta(NMP)$的dp转移对吧。
	\subsection{100pts}
	我们不难发现每一次$dp$的转移都是一样的,所以直接矩阵快速幂优化即可。
	\section{新生舞会(ball)}
	\subsection{100pts}
	没什么好说的,一个比较裸的0/1分数规划,后面求最大值直接用费用流即可。
	
	大概记录一下0/1分数规划吧。
	
	考虑题目要最大化一个分式,我们不妨定义这个分式为$\frac{A}{B}=C$。
	
	那么此时我们要最大化的就是$C$的值,所以我们可以把$C$换过去,大致是$A-B*C=0$。
	
	然后如果上述式子$\ge 0$,那么$C$的值就可以变大,反之就要变小。
	
	所以就可以二分完结了。
	\section{硬币游戏(game)}
	\subsection{100pts}
	考虑令$p_i$表示第$i$个人赢的概率,那么显然我们可以得到。
	
	$p_i+\sum_{j=1}^np_j(\sum_{k=1}^{m-[i==j]}[prefix(i,k)=suffix(j,k)]\frac{1}{2^{m-k}}) = \frac{1}{2^m}$
	
	然后高斯消元即可。
	\section{相关分析(relative)}
	\subsection{100pts}
	直接线段树维护所有你需要的东西即可。
	
	分块大法好。
	
	注意会爆$long\ long$,有些东西的维护要用$double$。
	\section{后话}
	完结撒花!=\_=
	
	可能今后还要再重新写一下这些题目,Fighting!
\end{document}
